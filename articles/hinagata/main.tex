\documentclass{word}
\usepackage{word-pdf}
\usepackage{wordlogo}
\usepackage{fancybox}
\usepackage{url}
\usepackage{listings}

\lstset{%
	basicstyle=\ttfamily,
	commentstyle=\textit,
	frame=trBL,
	numbers=left,
	breaklines=true,
}

\articletitle{記事の書き方}
\articleauthor{hoge}
\articleheader{ヘッダ}

\begin{document}
\makearticletitle
\section{はじめに}
\subsection{p\LaTeX を使う}

\subsubsection{macOS・Linux}

$article\_name$は適当な名前として、以下のようなコマンドでブランチを分けましょう。

\begin{lstlisting}[mathescape]
git submodule update --init
git checkout -b personal/$username$/$article\_name$
cd ./articles
cp -r ./hinagata ./my-article-name
cd ./my-article-name
autoconf
./configure
\end{lstlisting}

\subsubsection{Windows}

\WORD クラスファイルはWindowsでもコンパイルすることができます。
次のように\lstinline|cmake|を使います。

\begin{lstlisting}[mathescape]
git submodule update --init
git checkout -b personal/$username$/$article\_name$
cd ./articles
cp -r ./hinagata ./my-article-name
cd ./my-article-name
cmake -DENABLE_LUATEX=OFF .
\end{lstlisting}
\subsection{Lua\LaTeX を使う}
\lstinline|./configure|のかわりに\lstinline|./configure --enable-luatex|としてください。
Windowsの場合は、\lstinline|cmake -DENABLE_LUATEF=OFF -DENABLE_XETEX=OFF .|のかわりに
\lstinline|cmake -DENABLE_LUATEX=ON -DENABLE_XETEX=ON .|としてください。

\subsection{Xe\LaTeX を使う}

\section{記事を書く}
記事を書いたら、\lstinline|make|コマンドでビルドできます。
Windowsの場合は\lstinline|cmake|コマンドでビルドします。

\subsection{macOS・Linux}

\begin{lstlisting}
git add *
make
\end{lstlisting}

\subsection{Windows}

\begin{lstlisting}
git add *
cmake --build .
\end{lstlisting}

これで\ovalbox{main.pdf}が生成されれば成功です。
あとは\ovalbox{main.tex}を編集すれば記事が出来ます。

\section{Gitサーバにpushする}

記事のキリの良いところで\lstinline|git push|するといいのですが、最初のpushの時には、
origin\footnote{ここではWORDのGitサーバである\url{gitolite.word-ac.net}のことです}%
に新しいブランチを登録する必要があります。それは以下のようにしましょう。

\begin{lstlisting}[mathescape]
git push origin personal/$username$/$article\_name$
\end{lstlisting}

pushを成功させた場合には、ビルドの結果がslack\footnote{\url{https://word-ac.slack.com}}の\#jenkinsチャンネルに流れます。
slackを見ていない場合は、\url{https://jenkins.word-ac.net/job/LaTeX/}および\url{https://gitiles.word-ac.net/}を見ると良いでしょう。

\section{トラブルシューティング}
\subsection{偶数頁}
編集作業をしていると、レイアウトの問題で偶数頁から開始していただくことがあります。
その場合の対処法は、
\lstinline|\documentclass|のオプションに\ovalbox{evenstart}をつけることで簡単にできます。

\begin{lstlisting}[language=TeX, mathescape]
\documentclass[evenstart]{word}
\end{lstlisting}

\section{他の問題について}
問題があればslackの\#latexチャンネルや、編集会議で聞くと良いでしょう。

bxjsarticleを基にした新クラスについて聞く場合、@hidaruma\footnote{\url{https://twitter.com/hid_alma1026}}までどうぞ。

\end{document}